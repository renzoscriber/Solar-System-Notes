\documentclass[a4paper]{article}

% Start preamble
\usepackage{fullpage} 
\usepackage{parskip} 
\usepackage{tikz} 
\usepackage{amsmath} 
\usepackage{hyperref}
\usepackage{amsmath,amssymb}
\usepackage{color}
\usepackage[version=4]{mhchem} 
\usepackage{bm}
\usepackage{verbatim}
\usepackage{subfig}
\usetikzlibrary{arrows,automata,positioning,backgrounds,calc,fadings,shapes}
\usetikzlibrary{decorations.shapes}
\usetikzlibrary{shapes,arrows,automata,positioning,fit,backgrounds,petri}
\usetikzlibrary{fadings,decorations.pathmorphing}
% End preamble

\title{Notes on the Solar System}
\author{Lauren Shriver}
\date{07/12/2018}

\begin{document}
	\maketitle
	\section{Miscellaneous}
	Two angles of measurement used by astronomers to describe a star's position in the sky (Note: both angles are relative to the celestial equator)
	\begin{itemize}
		\item \textbf{Declination} = similar to longitude, but is projected on the celestial sphere
		\item \textbf{Right ascension} = known as the "hour angle" because it accounts for time of day and Earth's rotation 
	\end{itemize}
	\section{Suspicious Observers Notes}
	\subsection*{Sun Series}
	\subsubsection*{Solar Wind Introduction | Sun Series 1}
			\begin{itemize}
				\item The sun exerts outward current called the \textbf{solar wind} that particles travel through from the center of our solar system towards its outer rims 
				\item The solar wind is primarily comprised of Hydrogen ions and other charged particles 
					\begin{itemize}
						\item These particles can interact to create neutral atoms 
						\item Moreover, every spectrum of elemental matter has been detected in the solar wind!
					\end{itemize}
				\item The ions in solar wind (mainly oxygen and nitrogen) are what generate the Aurora's at our Earth's poles 
				\item Up until 2016, older satellite equipment called the Solar Dynamics Observatory (SDO;need to see if they have better data available) took measurements of particle density, speed, and temperature in the solar wind 
			\end{itemize}
	\subsubsection*{Layers of the Sun | Sun Series 2}
			\begin{itemize}
				\item SDO Data Highlights 
				\begin{itemize}
					\item SDO/AIA $94 \mathbf{\AA}$ images detected EUV/XRAY output from the Sun and ionized Fe
					\item SDO/AIA $193 \mathbf{\AA}$ images detected EUV output from the Sun and ionized Fe
					\item SDO/AIA $304 \mathbf{\AA}$ images detected EUV output from the Sun and ionized He 
				\end{itemize}
				\item \textbf{Sun spots} are probably large patches of Fe (generating magnetic fields)
				\item SDO/AIA $211 \AA$ images showed \textbf{coronal holes} (look like huge black patches)				
				\begin{itemize}
					\item We don't detect closed magnetic fields here
					\item In fact, we don't detect anything here
					\item This is because these patches are where high-energy, one-way currents (i.e., the solar wined)  travel outward into the solar system faster than we can take image samples
					\item Aside: a time-series FFT does not yield accurate frequency data for frequencies higher about equal to or higher than the sampling frequency 
				\end{itemize}
				\item Above the corona, there are plasma filaments "defying" gravity via magnetic forces 7
			\end{itemize}			
	\subsection*{Miscellaneous Videos}
	\subsubsection*{What Are Cosmic Rays? (7/15/18)}
	\begin{itemize}
		\item Modern maximum of cosmic rays was reached mid-2018
		\item All forecasts indicate increasing cosmic rays for the remainder of the century 
		\item What are cosmic rays?
			\begin{itemize}
				\item Cosmic rays are spend a lot of time traveling through space (can think of them as "space voyager" particles) at speeds near that of the speed of light 
				\item Produced in high-energy events (e.g., SuperNova blasts, cosmic jets, wild-orbital binary stars, etc.) accelerate these particles such high speeds 
				\item Cosmic rays primarily consist of protons and atomic nuclei stripped of their \ce{e^-}s (by the event that created them)
				\item Specifically, most of the particles traveling in cosmic rays are one of the following:
					\begin{itemize}
						\item \ce{1 p^+} (+1 charge)
						\item \ce{H} 	(+1 charge)
						\item \ce{He}	(+2 charge)
						\item \ce{C}	(+6 charge)
						\item \ce{Mg}	(+12 charge)
						\item \ce{Fe}	(+26 charge)
					\end{itemize} 
			\end{itemize}
		\item Every $\mathrm{m^2}$ every $\mathrm{s}$ of Earth's atmosphere is being hit by cosmic rays  
			\begin{itemize}
				\item When particles do hit the upper atmosphere (or the mantle in the case of super-high-energy cosmic rays), a cascade effect occurs that produces protons, electrons, positrons, neutrons, x-rays, and gamma-rays
				\item It is important to remember that this is constantly occurring everywhere on Earth's atmosphere
				\item This also results in "air travel" radiation (estimate we get a "dental x-ray" amount of radiation on an average flight) 
				\item Study detected the following radiation at airplane heights:
					\begin{itemize}
						 \item EM Waves
						 	\begin{itemize}
						 		\item Gamma rays;  very high power;   were detected
						 		\item X-rays;   very high power;    were detected 
						 	\end{itemize}
						 \item Charged Particles 	
						 	\begin{itemize}
						 		\item Cosmic rays;   very high;   were not detected
						 		\item Electrons;   low to high;   were not detected
						 		\item Muons;    low to high;     were not detected
						 		\item Neutrons;    low to high;  were detected 
						 	\end{itemize}
					\end{itemize}
			\end{itemize}
	\end{itemize}
	\section{QuickStudy Science Solar System Notes}
		\subsection*{What does the Sun orbit?}
		The Sun (and thus the entire solar system), orbits about the center of the Milky Way galaxy
			\begin{itemize}
				\item The galactic center is about 26,000 light years away from the Sun
				\item The Sun travels along an elliptical orbit at about 225,308 m/s 
				\item The Sun's orbit rate is once every 225 million years (this is called the \textbf{cosmic year})
			\end{itemize}
\end{document}
